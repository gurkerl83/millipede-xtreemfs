The Metadata and Replica Catalog (MRC)\index{MRC} is responsible for the management of all metadata in an XtreemFS installation. Core tasks of the MRC are the management of volumes and directory trees, storage of file and directory metadata and access control enforcement.

\subsubsection {Architecture}

Aside from the ONC RPC\index{ONC RPC} server that listens for incoming client requests, the MRC\index{MRC} architecture comprises two core components: the processing stage and the database backend. Each request received by the ONC RPC\index{ONC RPC} server is parsed and forwarded to the processing stage, which executes the respective file system logic. Any data that needs to be retrieved or modified during file system logic execution is stored in a database backend.

\subsubsection{Processing Stage}

The MRC\index{MRC} interface consists of multiple so-called \emph{operations}. Each operation relates to an implementation of the logic for the execution of a certain request. There are operations e.g.\ for opening files, reading directory content, creating volumes, and the like. Operations are named and parametrized similar to their corresponding POSIX\index{POSIX} calls. To circumvent locking issues in the underlying database, operation execution is serialized for each volume, i.e.\ no more than one thread may execute operations on a certain volume at the same time.

All operations have a similar composition. First, authorization checks are performed, in order to find out whether the user on behalf of whom the request was sent has sufficient permissions to execute the operation. In case of a positive result, the operation logic is executed. Operation logic execution may involve an arbitrary number of accesses to the underlying database backend. A \texttt{readdir} request will e.g.\ result in a database lookup for the content of a directory, a \texttt{setxattr} request will cause an extended attribute of a file to be added in the database.

A detailed description of the interface to the MRC\index{MRC} including all operations is given in Sec.\ \ref{sec:mrc_interface}.

\subsubsection{Database Backend}

\newenvironment{mappingTable}[1]{\fontfamily{pcr}\begin{center}\begin{footnotesize}\begin{tabular}{|l|} \hline \textnormal{\textbf{\small{#1}}} \\ \hline \hline \\}{\end{tabular}\end{footnotesize}\end{center}\fontfamily{default}}

\newenvironment{internalMappingTable}[1]{\tabularx{13.3cm}{|m{2.5cm}|m{1.5cm}|X|} \multicolumn{3}{l}{\textnormal{\textbf{\small{#1}}}} \\ \multicolumn{3}{l}{} \vspace{-0.3cm} \\ \hline \textnormal{\textbf{Element}} & \textnormal{\textbf{\# Bytes}} & \textnormal{\textbf{Description}} \\ \hline }{\endtabularx}

The database backend is accessed at record level, i.e.\ at a granularity of single key-value pairs. The creation of a new file could e.g.\ require several record modifications, since a file metadata object needs to be inserted in the database, a link to the parent directory needs to be established, time stamps of parent directories need to be updated, and so forth. Multiple such records can be combined in an insert group, which causes the insertion of a new set of records to take place in a single step, i.e.\ atomically. 

The database backend implementation is decoupled from the remaining MRC\index{MRC} code via an interface, which gives developers the opportunity to implement their own database bindings. The currently used implementation is based on BabuDB. A BabuDB instance may comprise multiple databases, which may in turn comprise multiple indices. Databases are identified by name strings, whereas indices of a database are serially numbered. Lookups and insertions are directed to single indices of a database; besides normal value lookups for keys, BabuDB supports queries for key prefixes, which provides the basis for an efficient lookup of consecutive key-value pairs.

A range of different indices are used to store XtreemFS metadata. How XtreemFS metadata is mapped to BabuDB indices will be described in the following.

\paragraph{Metadata for Volume Management}

Volume metadata is stored in a database named \texttt{V}. It is arranged in the following indices:

\begin{footnotesize}
\begin{center}
\begin{tabularx}{\linewidth}{|r|l|X|}
\hline
\textbf{\#} & \textbf{Name} & \textbf{Description} \\
\hline
0 & Volume ID Index & Maps a volume UUID to a volume metadata entity. \\
\hline
1 & Volume Name Index & Maps a volume name to a volume ID. \\
\hline
\end{tabularx}
\end{center}
\end{footnotesize}

\begin{mappingTable}{Volume ID Index}

\begin{internalMappingTable}{key}
volumeID & var & the volume ID string \\
\hline
\end{internalMappingTable}

\\
\\

\begin{internalMappingTable}{value}
fileAccPolID & 2 & the file access policy ID for the volume \\ \hdashline
osdPolID & 2 & the OSD selection policy ID for the volume \\ \hdashline
offsVolName & 2 & the offset position of the 'volName' element, relative to the offset of the buffer's first byte\\ \hdashline
offsPolArgs & 2 & the offset position of the 'osdPolArgs' element, relative to the offset of the buffer's first byte\\ \hdashline
volID & var & the volume's UUID string\\ \hdashline
volName & var & the volume's name string\\ \hdashline
osdPolArgs & var & the volume's OSD selection policy argument string\\ \hline
\end{internalMappingTable}

\\
\\
\hline

\end{mappingTable}

\begin{mappingTable}{Volume Name Index}

\begin{internalMappingTable}{key}
volName & var & the volume name string \\ \hline
\end{internalMappingTable}

\\
\\

\begin{internalMappingTable}{value}
volId & var & the volume's UUID string \\ \hline
\end{internalMappingTable}

\\
\\
\hline

\end{mappingTable}

\paragraph{Metadata for Files and Directories}

File system metadata of a volume is stored in a BabuDB database with a name equal to the volume's UUID. Various indices are used to manage metadata pertaining to files and directories, which will be described in the following tables. Indices have been designed with the following goals in mind:
\begin{itemize}
 \item Lookups performed by frequently invoked operations should be as fast as possible, like metadata lookups for a given directory path.
 \item Database records that are frequently updated should include as little unchanged data as possible.
 \item Frequently performed database updates should be fast, i.e.\ involve as little index insertions as possible.
 \item Indices should contain as little redundancy as possible, in order to minimize database size and memory footprint.
\end{itemize}

With the aforementioned goals in mind, we decided to have a primary index for the primary metadata of files, which maps a key essentially consisting of a parent directory ID and a file name hash to a value that contains a metadata record. This way, BabuDB prefix lookups for parent directory IDs can be used to efficiently retrieve contents of a directory, while normal lookups can be used to retrieve metadata for a single file. Since POSIX\index{POSIX} requires support for hard links, i.e.\ different directory entries pointing to the same metadata, and some operations require a retrieval of file metadata by means of file IDs, we decided to maintain a secondary index that allows a retrieval of metadata by means of a file ID. Other indices are used to store extended attributes and access control lists.

\begin{footnotesize}
\begin{center}
\begin{tabularx}{\linewidth}{|r|l|X|}
\hline
\textbf{\#} & \textbf{Name} & \textbf{Description} \\
\hline
0 & File Index & Stores primary metadata for a directory entry. Values in the index may be of different kinds:
\begin{itemize}
 \item \emph{frequently changed metadata} - encapsulates all metadata that is frequently modified, such as time stamps or file sizes
 \item \emph{rarely changed metadata} - encapsulates all metadata that is infrequently changed, such as file names, access modes, or ownership of a file
 \item \emph{replica location metadata} - encapsulates X-Location lists of files
 \item \emph{hard link targets} - in case additional hard links exist for one file, the value is a hard link target, i.e.\ a key in the File ID index. Lookups to file metadata will then be performed in two steps: first, a lookup in the File Index will be performed, in order to retrieve the hard link target; then, metadata will be looked up in the File ID Index.
\end{itemize}
\\ 
\hline
1 & XAttr Index & Contains any extended attributes of files and directories. This includes Softlink targets and default striping policies, since they are mapped to extended attributes.\\
\hline
2 & ACL Index & Contains access control list entries of all files.\\
\hline
3 & File ID Index & The file ID index is used to retrieve file metadata by means of its ID. If no hard links have been created to a file, the file ID will be mapped to a key in the file index, for which the metadata will have to be retrieved with a second lookup. Such a mapping is necessary for some operations that are based on file IDs instead of path names. If hard links have been created, the file ID will be directly mapped to the three different types of primary file metadata (i.e.\ rarely and frequently changed metadata, as well as replica locations). In this case, the file's entries in the file index point to the corresponding prefix key in the file ID index.\\
\hline
4 & Last ID Index & Contains a single key-value pair that maps a static key to the last file ID that has been assigned to a file. The index ensures that new file IDs are assigned to newly created files or directories.\\
\hline
\end{tabularx}
\end{center}
\end{footnotesize}


\begin{mappingTable}{File Index}

\begin{internalMappingTable}{key}
parentID & 8 & file ID of the parent directory \\ \hdashline
fileNameHash & 4 & a hash value of the file name \\ \hdashline
type & 1 & type of metadata (0=frequently changed metadata, 1=rarely changed metadata, 2=replica locations, 3=hard link targets)\\ \hdashline
collCount & 2 & counter that is incremented with each collision of file name hashes - will be omitted unless multiple file names in the directory have the same hash values\\ \hline
\end{internalMappingTable}

\\
\\

\begin{internalMappingTable}{value, type = 0}
fcMetadata & 20\slash12\slash8 & frequently changed metadata associated with the file (see 'fcMetadata' definition), 20 bytes for files \& symlinks, 12 for directories, 8 for hard link targets\\ \hline
\end{internalMappingTable}

\\
\\

\begin{internalMappingTable}{value, type = 1}
rcMetadata & var & rarely changed metadata associated with the file (see 'rcMetadata' definition)\\ \hline
\end{internalMappingTable}

\\
\\

\begin{internalMappingTable}{value, type = 2}
xLocList & var\slash8 & the replica location list associated with the file (see 'xLocList' definition), variable length for files \& directories, 8 for hard link targets\\ \hline
\end{internalMappingTable}

\\
\\
\hline

\end{mappingTable}


\begin{mappingTable}{XAttr Index}

\begin{internalMappingTable}{key}
fileID & 8 & the ID of the file to which the extended attribute has been assigned\\ \hdashline
ownerHash & 4 & a hash value of the attribute's owner\\ \hdashline
attrNameHash & 4 & a hash value of the attribute name\\ \hdashline
collCount & 2 & counter that is incremented with each collision of (ownerHash, attrNameHash) pairs - will be omitted unless different attributes are hashed to the same such pair\\ \hline
\end{internalMappingTable}

\\
\\

\begin{internalMappingTable}{value}
offsKey & 2 & the offset position of the 'attrKey' element, relative to the offset of the buffer's first byte\\ \hdashline
offsValue & 2 & the offset position of the 'attrValue' element, relative to the offset of the buffer's first byte\\ \hdashline
attrOwner & var & the user ID of the attribute's owner\\ \hdashline
attrKey & var & the attribute key\\ \hdashline
attrValue & var & the attribute value\\
\hline
\end{internalMappingTable}

\\
\\
\hline

\end{mappingTable}


\begin{mappingTable}{ACL Index}

\begin{internalMappingTable}{key}
fileID & 8 & the ID of the file to which the extended attribute has been assigned\\ \hdashline
entityName & var & the name of the entity associated with the ACL entry\\ \hline
\end{internalMappingTable}

\\
\\

\begin{internalMappingTable}{value}
rights & 2 & the access rights for the entity\\ \hline
\end{internalMappingTable}

\\
\\
\hline

\end{mappingTable}


\begin{mappingTable}{File ID Index}

\begin{internalMappingTable}{key}
fileID & 8 & the ID of the file\\ \hdashline
type & 1 & type of metadata (0=frequently changed metadata, 1=rarely changed metadata, 2=replica locations, 3=hard link targets)\\ \hline
\end{internalMappingTable}

\\
\\

\begin{internalMappingTable}{value, type = 0}
fcMetadata & 20\slash12 & frequently changed metadata associated with the file (see 'fcMetadata' definition), 20 bytes for files \& symlinks, 12 for directories\\ \hline
\end{internalMappingTable}

\\
\\

\begin{internalMappingTable}{value, type = 1}
rcMetadata & var & rarely changed metadata associated with the file (see 'rcMetadata' definition)\\ \hline
\end{internalMappingTable}

\\
\\

\begin{internalMappingTable}{value, type = 2}
xLocList & var & the replica location list associated with the file (see 'xLocList' definition)\\ \hline
\end{internalMappingTable}

\\
\\

\begin{internalMappingTable}{value, type = 3}
parentID & 8 & the ID of the parent directory in which the metadata for the file is stored\\ \hdashline
fileName & var & the file name in the parent directory\\ \hline
\end{internalMappingTable}

\\
\\
\hline

\end{mappingTable}


\begin{mappingTable}{Last ID Index}

\begin{internalMappingTable}{key}
'*' & 1 & the only key in the table\\
\hline
\end{internalMappingTable}

\\
\\

\begin{internalMappingTable}{value}
lastFileID & 8 & the last ID that has been previously assigned to a file or directory\\ \hline
\end{internalMappingTable}

\\
\\
\hline

\end{mappingTable}


Data types referenced in the index descriptions above are listed in the following:

\begin{mappingTable}{frequentlyChangedMetadata}

\begin{internalMappingTable}{files}
atime & 4 & file access time stamp in seconds since 1970\\ \hdashline
ctime & 4 & file metadata change time stamp in seconds since 1970\\ \hdashline
mtime & 4 & file content modification time stamp in seconds since 1970\\ \hdashline
size & 8 & file size in bytes\\ \hline
\end{internalMappingTable}

\\
\\

\begin{internalMappingTable}{directories}
atime & 4 & file access time stamp in seconds since 1970\\ \hdashline
ctime & 4 & file metadata change time stamp in seconds since 1970\\ \hdashline
mtime & 4 & file content modification time stamp in seconds since 1970\\ \hline
\end{internalMappingTable}

\\
\\
\hline

\end{mappingTable}


\begin{mappingTable}{rarelyChangedMetadata}

\begin{internalMappingTable}{files}
type & 1 & the type of the entry (0=file, 1=directory)\\ \hdashline
id & 8 & file ID\\ \hdashline
mode & 4 & POSIX\index{POSIX} access mode\\ \hdashline
linkCount & 2 & number of hard links to the file\\ \hdashline
w32attrs & 8 & Win32-specific attributes\\ \hdashline
epoch & 4 & current truncate epoch\\ \hdashline
issEpoch & 4 & last truncate epoch that has been issued\\ \hdashline
readOnly & 1 & a flag indicating whether the file is suitable for read-only replication\\ \hdashline
offsOwner & 2 & offset position of the 'owner' element, relative to the offset of the buffer's first byte\\ \hdashline
offsGroup & 2 & offset position of the 'group' element, relative to the offset of the buffer's first byte\\ \hdashline
fileName & var & name of the file\\ \hdashline
owner & var & user ID of the file's owner\\ \hdashline
group & var & group ID of the file's owner\\ \hline
\end{internalMappingTable}

\\
\\

\begin{internalMappingTable}{directories}
type & 1 & the type of the entry (0=file, 1=directory)\\ \hdashline
id & 8 & file ID\\ \hdashline
mode & 4 & POSIX\index{POSIX} access mode\\ \hdashline
linkCount & 2 & number of hard links to the file\\ \hdashline
w32attrs & 8 & Win32-specific attributes\\ \hdashline
offsOwner & 2 & offset position of the 'owner' element, relative to the offset of the buffer's first byte\\ \hdashline
offsGroup & 2 & offset position of the 'group' element, relative to the offset of the buffer's first byte\\ \hdashline
fileName & var & name of the directory\\ \hdashline
owner & var & user ID of the directory's owner\\ \hdashline
group & var & group ID of the directory's owner\\ \hline
\end{internalMappingTable}

\\
\\
\hline

\end{mappingTable}


\begin{mappingTable}{xLocList}

\begin{internalMappingTable}{xLocList}
version & 4 & version of the X-Locations list\\ \hdashline
replCount & 4 & number of replicas in the list\\ \hdashline
offsUpdPol & 4 & offset position of the 'updPol' element, relative to the offset of the buffer's first byte\\ \hdashline
offs1 \linebreak \dots \linebreak offsN & 4 \linebreak \dots \linebreak 4 & offset positions for all replicas, relative to the offset of the buffer's first byte\\ \hdashline
xLoc1 \linebreak \dots \linebreak xLocN& var \linebreak \dots \linebreak var & replicas in the X-Locations list\\ \hdashline
updPol & var & update policy string that describes how replica updates will be propagated\\ \hline
\end{internalMappingTable}

\\
\\

\begin{internalMappingTable}{xLoc}
offsOsdList & 2 & offset position of the 'osdList' element, relative to the offset of the buffer's first byte\\ \hdashline
strPol & var & striping policy associated with the replica\\ \hdashline
osdList & var & list of all OSDs\index{OSD} for the replica\\ \hline
\end{internalMappingTable}

\\
\\

\begin{internalMappingTable}{strPol}
stripeSize & 4 & size of a single stripe (=object) in kB\\ \hdashline
width & 4 & number of OSDs\index{OSD} for the striping\\ \hdashline
pattern & var & string containing the striping pattern\\ \hline
\end{internalMappingTable}

\\
\\

\begin{internalMappingTable}{osdList}
osdCount & 2 & number of OSDs\index{OSD} in the list\\ \hdashline
offsOSD1 \linebreak \dots \linebreak offsOSDn & 2 \linebreak \dots \linebreak  2 & offset positions for all OSD UUIDs, relative to the offset of the buffer's first byte\\ \hdashline
osdUUID1 \linebreak \dots \linebreak osdUUIDn & var \linebreak \dots \linebreak var & UUIDs of all OSDs in the list\\ \hline
\end{internalMappingTable}

\\
\\
\hline

\end{mappingTable}
