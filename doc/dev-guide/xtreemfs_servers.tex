All XtreemFS servers (DIR\index{DIR}, MRC\index{MRC} and OSD\index{OSD}) are written in Java and employ an event-based staged design. In addition to the common architecture, they also share the basic libraries like RPC server and client or memory management.

In our design, a stage has one or more threads to do the work. Usually, a stage is used for processing which is blocking (e.g. I/O operations) or consumes larger amounts of CPU time (e.g. checking signatures). Each stage receives requests (events), processes them asynchronously and passes the result to a callback. Operations are the ``glue'' between the stages. For each client request or internal event, there is an Operation class which implements the logic of the call.

All servers also use a custom method of memory management. To avoid excessive data copying to and from the Java VM, we use direct ByteBuffers which represent raw memory on the heap. These direct ByteBuffers are not managed by the Java garbage collector and excessive allocation and freeing of them causes severe performance problems. To overcome this problem and to reduce overall memory consumption, we use a concurrent BufferPool to allocate ByteBuffers. In addition, we use a wrapper class (called ReusableBuffer) which implements reference counting. It also ensures that ReusableBuffers which have been returned to the pool cannot be used anymore.

The ReusableBuffers must be freed (i.e. returned to the BufferPool) after using them. Failing to do so will cause an error message to be printed on finalization which should help to detect memory leaks. Setting \texttt{Buffer\-Pool.record\-Stack\-Traces} to \texttt{true} will add a full stack trace of the allocation to the error message which is useful to locate memory leaks. This option is only for debugging and should not be used for production due to the performance penalty of recording stack traces on each allocation.

The ONC RPC\index{ONC RPC} server and client are used by all three servers as well. Both are implemented using Java's non-blocking network IO NIO and can be used with or without SSL.

The DIR\index{DIR} and MRC\index{MRC} also use an external key-value store called BabuDB (see http://babudb.googlecode.com) to persistently store information. How it is used and how the data is stored in BabuDB is described in the DIR\index{DIR} and MRC\index{MRC} sections, respectivly.